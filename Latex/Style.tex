


\usepackage{xspace}

%\setkomafont{sectioning}{\rmfamily} %<-- für Überschriften in "`Roman"'-Schriftart





%---------------------------------------------
% Mehr Möglichkeiten bei Tabellen:
\usepackage{array}
\usepackage{multirow}
\usepackage{arydshln}

 
\usepackage[T1]{fontenc}
\usepackage{lmodern}
\usepackage[utf8]{inputenc}		%deutsche Silbentrennung,
% \usepackage[ngerman]{babel}   %neue Rechtschreibung mit ngerman
 \usepackage[english]{babel}

%\usepackage{ae,aecompl} % verhindert Bitmap-Schriften im PDF
%\usepackage{times} % verwendet Postscript-Schriftarten

%---------------------------------------------
% Zeilen-Abstands Paket laden
%\usepackage{setspace}
%            \onehalfspacing                    % oder
%            \doublespacing                     % oder
%            \singlespacing                      % oder
%            \begin{spacing}{2.3}
%            ...
%            \end{spacing}





%------------------------------------------
% Paket für schöne Tabellen
\usepackage{booktabs}
\usepackage{fancyvrb}
\usepackage{supertabular}
\usepackage{multirow}		% Paket, um Tabellenzellen vertikal zu verbinden

%---------------------------------------------
% Graphics laden für Grafiken
\usepackage{caption}

\usepackage{graphicx, geometry, psfrag}
\usepackage{float} %for figur possitioning with [H]
\usepackage{subfig}
\usepackage{siunitx}



\usepackage[section]{placeins} %sets a barrier for float figures in next section
% Bilder sollen "Abb." heißen (nicht "Abbildung")
%\addto\captionsngerman{
%  \renewcommand{\figurename}{Abb.}	% "Abb." statt "Abbildung" schreiben
%  \renewcommand{\tablename}{Tab.}	% "Tab." statt "Tabelle" schreiben
%		}


%---------------------------------------------
% Größe einer beschreibbaren Seite
\geometry{  body={15.5cm, 23cm}, top=3cm, left=3cm }

% Pakete um fertige PDF-Seiten einzubinden
\usepackage{eso-pic}
\usepackage{pdfpages}


%Liefert viele Tools für Math-Mode: 
\usepackage{amsmath}
\usepackage{amsthm}		%Ergänzung zu "amsmath"
\usepackage{amssymb}	% Symbole

% Paket für Rahmen
\usepackage{framed}


	
% \usepackage{chngcntr}              %für durchlaufendes Bildernummerieren
% \counterwithout{figure}{section}

%---------------------------------------------
% Index-Modul laden
%\makeindex
%\usepackage{makeidx}
%\makeglossary

%---------------------------------------------
% Landscape-Paket, um einzelne Teile auf Querformat zu setzen
\usepackage{lscape}
% Verwendung:
%	\begin{landscape}
%      ...
%	\end{landscape}

\usepackage{rotating}

%---------------------------------------------
% Fußnoten am Text ausgerichtet
%\usepackage[flushmargin]{footmisc} %Fußnoten am Text ausgerichtet


%------------------------------------------
%Erlaubt das Grad-Celsius Zeichen mit direkt via \grad zu setzen
\newcommand{\grad}{\ensuremath{^\circ}}


%------------------------------------------
% Schriftgröße des Lit-Verz. ändern
%\newcommand{\bibfont    }{\footnotesize}
% Literaturverzeichnis nach DIN-Norm
\usepackage{natbib}

% vertikalen Abstand ändern
\setlength\bibsep{1.4mm}
\usepackage{color}
\definecolor{darkblue}{rgb}{0,0,.7}



% Tiefe des Inhaltsverzeichnis angeben, Standard: 2
%\setcounter{tocdepth}{2}




%---------------------------------------------
% PDF-Paket und Hypperref laden (zum Einstellen von Autor, Titel, .... in PDF-Infos)
% 	muss am Ende vom Header stehen

\usepackage{wrapfig}
\usepackage{textcomp}
\usepackage{setspace}
\usepackage{fancyhdr}

\newcommand{\changefont}[3]{
\fontfamily{#1} \fontseries{#2} \fontshape{#3} \selectfont}

\changefont{pcr}{m}{n}
\usepackage[
        pdftex,
        %a4paper,
        %backref,
        %pagebackref,
        bookmarks,
        bookmarksopen=false,
        bookmarksnumbered=false,
        pdfsubject={Master Thesis},
        pdfauthor={Benjamin Hollmach},
        pdftitle={TITEL},
        pdfkeywords={KEYWORDS},
        colorlinks=true, 
        linkcolor= black,		%darkblue
        %menucolor=darkblue,
        citecolor= black,		%darkred
        urlcolor= darkblue, % black,
		pdfcreator={Name},
]{hyperref} % Für Print-Version das "hyperref" auskommentieren
